\documentclass[titlepage]{article}
\usepackage[utf8]{inputenc}
\usepackage{amsmath}
\usepackage{mismath}
\usepackage{graphicx}
\usepackage{listings}
\usepackage{float}
\usepackage{hyperref}
\usepackage{geometry}
 \geometry{
 a4paper,
 top=30mm,
 }
\lstset{
  basicstyle=\fontsize{11}\selectfont
}

% PACKAGES X DIBUIXAR ARBRES
\usepackage[T1]{fontenc}
\usepackage[linguistics]{forest}
%%%%%%%%%%%%%%%%%%%%%%%%%%%%

\title{Anàlisi de quatre mètodes de resolució de la Super Sopa}
\author{Aina Gomez, Paula Muñoz, Jordi Palomera, Neus Mayol}
\date{Octubre 2022}


\begin{document}

\maketitle
\clearpage
\renewcommand{\contentsname}{Índex}
\tableofcontents
\clearpage

\section{Introducció}
\par
Moltes vegades relacionem els algorismes amb la modernitat, l'ús de les noves tecnologies o les xarxes socials. I sí, és veritat, l'algorismia juga un paper important en aquests camps, 
però si retrocedim temporalment veurem que ja fa centenars i milers d'anys que fem servir aquests procediments. Podem dir que els algorismes sempre ens han donar el camí per trobar la solució a tot tipus de problemes quotidians d'ordenació, de càlcul, de lògica etc.\newline\par

\par
És més, independentment de la seva dificultat, cost temporal o de memòria, tots tenen una cosa en comú: mecanitzar i agilitzar tasques. És just en aquest moment —quan les tasques es poden mecanitzar— que podem deixar l'execució de l'algorisme a mans d'un ordinador, que resoldrà problemes com el que tracta aquest escrit de manera fàcil i ràpida.

\section{Formulació del problema}
El problema presenta una sopa de lletres de mida \begin{math}n\times n\end{math}, les paraules de la qual poden disposar-se en qualsevol de les vuit direccions. A més, una paraula pot tenir canvis de direcció entremig i solapar-se amb una altra paraula, però mai es sobreposarà amb ella mateixa.\newline\par
Com el lector ja haurà pogut deduir, el principal problema plantejat és la cerca de paraules dins la sopa de lletres. Les paraules que s'han de buscar a la sopa venen determinades per un diccionari donat. Per assegurar-se que es trobaran paraules a la sopa, s'hi introduiran alguns mots del diccionari al moment de  crear-la. Ara bé, al moment de resoldre-la, s'hauran de trobar tots els mots que hi hagi al diccionari, no només els que s'han introduït expressament.\par


\section{Metodologia}
Per comoditat i adequació amb el  que es demana a l'enunciat, s'ha decidit programar els algorismes en C++. Durant el treball s'aniran ensenyant els algorismes segons aquest llenguatge, la resta del codi serà visible dins el repositori de \textit{GitHub} enllaçat al final de l'escrit.\newline\par
Per tal de resoldre la sopa de lletres, l'enunciat planteja quatre maneres d'emmagatzemar el diccionari de paraules donat. Per tal d'organitzar millor les tasques a l'hora de programar, s'ha decidit crear una classe per a cada una de les maneres seguents i que podreu trobar explicades amb detall més endavant.
\begin{itemize}
  \item Amb un vector ordenat (\textit{Sorted Vector}).
  \item Amb un \textit{trie}.
  \item Amb un filtre de Bloom.
  \item I amb una taula de \textit{hash} amb \textit{double hashing}.
\end{itemize}

També, hem creat una cinquena classe anomenada \textit{SuperSopa}, on hi guardem les funcions principals de la sopa de lletres. Dins d'aquesta classe, podem trobar diferents procediments i funcions com, per exemple, omplir la sopa o bé per resoldre-la.\par

\clearpage
\section{Vector ordenat (o \textit{Sorted Vector})}
L'estructura de dades més simple i coneguda. En aquest cas s'emmagatzemarà cada paraula dins una posició del vector. Per tant, i mirant-ho des de termes del llenguatge C++, sempre es parlarà d'un vector de paraules (o \textit{string}).\newline\par

El vector ha de ser ordenat, les paraules estaran organitzades alfabèticament tal i com els diccionaris. L'enunciat no garanteix que les paraules del diccionari donat estiguin en ordre alfabètic. És per això que, abans que res, s'haurà de proucurar que estiguin ben ordenades. D'aquesta manera, el vector s'estructurarà tal i com representa la \textit{Figure \ref{fig:sortedVector}}, on s'han fet servir paraules d'exemple ordenades alfabèticament.


\begin{center}
    \begin{figure}[h!]
        \begin{center}
            \begin{tabular}{ |c|c|c|c|c|c|c| } 
             \hline
                 ... & any & bou & plor & rosa & urna & ...\\ 
             \hline
            \end{tabular}
        \end{center}
            \caption{Exemple de vector ordenat}
            \label{fig:sortedVector}
    \end{figure}
\end{center}

I és que de fet, el vector és el principal atribut de la classe \textit{SortedVector}, definida al fitxer \textit{diccSortedVector.cc} i al seu fitxer de capçaleres corresponent: \textit{diccSortedVector.hh}. L'altre atribut de la classe és un \textit{map} anomenat \textit{trobades}, la clau del qual és una paraula que és solució de la sopa i s'aparella amb un enter corresponent al nombre de vegades que hi apareix.
\newline\par

\subsection{Funcionament de la classe i anàlisi del seu cost}

Tal com s'ha explicat anteriorment, abans de començar a solucionar la sopa, hem de construir l'estructura de dades adequada per emmagatzemar el diccionari, en aquest cas un vector de paraules ordenades alfabèticament. Tenint l'estructura creada, ja es podrà procedir a la resolució de la sopa.

\subsubsection{Construcció de l'estructura}
En primer lloc i des del fitxer \textit{experiment\_vector.cc}, es crearà una instància de la classe \textit{SortedVector}. Seguidament s'afegirà el diccionari de paraules al vector cridant la funció \textit{afegir} de la classe \textit{SortedVector}. L'únic paràmetre de la funció \textit{afegir} és el diccionari que, al ser també un vector de paraules, només caldrà ordenar i emmagatzemar a l'atribut de la classe. Per ordenar el vector s'han considerat diversos algorismes, finalment s'ha decidit utilitzar l'ordenació \textit{Mergesort}.

\paragraph{\textit{Mergesort}} 
Aquest procediment és un dels mètodes més eficients d'ordenació. Pertany al grup dels algorismes de dividir i vèncer (o \textit{divide and conquer}, en anglès), que agrupa tots aquelles funcions que divideixen el problema en subproblemes fàcils de resoldre per tal d'acabar resolent el problema en general. \newline\par

És per això que el \textit{Mergesort} comença dividint el vector de string en dos subvectors, i així recursivament fins a obtenir  vectors de mida 1. Quan els subvectors tenen mida 1 es sap del cert que estaran ordenats, ja que són vectors d'un sol element. S'ha arribat a un problema fàcil de resoldre i per tant, l'algorisme podrà començar a solucionar la totalitat del problema.\newline
Tot aquest procés es pot veure dins el procediment \textit{mergesort}, que pertany a la classe \textit{SortedVector}.\newline\par


\begin{lstlisting}[language=C++]
void SortedVector::mergesort(int l, int r) {
    if (l < r) {
        int m = (l + r) / 2;
        mergesort(l,m);
        mergesort(m+1,r);
        merge(l,r,m);
    }
}
\end{lstlisting}
Tal com mostra el codi, les crides a \textit{mergesort} s'encarreguen de la divisió i la funció \textit{merge} s'ocupa de l'ordenació dels subvectors i la fusió d'aquests. La funció \textit{merge} sempre rebrà un vector on cada meitat està ordenada independentment de l'altra. Es pot assegurar que sempre serà així perquè la ordenació comença en subvectors de mida 1 —ordenats de per sí— i va fusionant subvectors ja ordenats i repetint el mateix procediment fins a obtenir el vector inicial ordenat.\newline
El fragment de codi a continuació realitza la tasca comentada: En primer lloc es creen dos vectors corresponents als subvectors que s'han d'organitzar. Per ordenar el vector es recorren ambdós subvectors linealment amb dos iteradors, a cada iteració es compararen els valors dels subvectors en la posició de cada iterador i el més petit dels dos valors es guarda al vector \textit{v}, que contindrà el vector ja ordenat. Seguidament s'augmenta el valor de l'iterador del subvector que contenia el valor més petit i torna a començar el bucle.\newline\par
Com que no podem assegurar que els dos subvectors tenen la mateixa mida, en acabar l'ordenació hi ha dos bucles que acaben de posar els valors del subvector de mida superior.

\begin{lstlisting}[language=C++]
void SortedVector::merge(int l, int r, int m) {
    int n1 = m - l + 1, n2 = r - m;
    
    // Dos vectors amb els valors de cada subvector:
    vector< string> v1(n1), v2(n2);
    for (int i = 0; i < n1; i++)
        v1[i] = v[l + i];
    for (int j = 0; j < n2; j++)
        v2[j] = v[m + 1 + j];

    // Ordenacio:
    int i = 0, j = 0, k = l;
    while (i < n1 and j < n2) {
        if (v1[i] <= v2[j]) {
            v[k] = v1[i];
            ++i;
        } else {
            v[k] = v2[j];
            ++j;
        }
        ++k;
    }

    // Ordenacio de les parts restants
    while (i < n1) {
        v[k] = v1[i];
        ++i; ++k;
    }
    while (j < n2) {
        v[k] = v2[j];
        ++j; ++k;
    }

}
\end{lstlisting}

Ja per acabar, és ben sabut que el cost de l'algorisme \textit{Mergesort} és $\Theta(n\log{}n)$ en tots els seus casos, ja que fa servir dues crides recursives que divideixen el vector per la meitat, i la part no recursiva té cost $\Theta(n)$.


\subsubsection{Resolució de la sopa}
Al tenir el vector ordenat alfabèticament, s'ha cregut que l'opció més eficient és utilitzar la cerca dicotòmica per resoldre la sopa. Tot i així s'ha decidit retocar la manera com s'implementa aquesta per adaptar-se al problema, i que no s'hagi de cercar dins de tot el diccionari cada vegada que es llegeix una nova lletra de la sopa. \newline\par
Per començar es recorrerà tota la sopa posició per posició. A cada posició es buscarà totes les solucions que comencen des d'allà. Per fer-ho s'entrarà a l'algorisme \textit{buscarParaula}.
\paragraph{\textit{buscarParaula}}
La idea de l'algorisme \textit{buscarParaula} pot semblar complicada d'entrada, però s'ha considerat que és un mètode eficient. S'ha vist que la cerca d'una paraula de la sopa al diccionari pot donar lloc a diferents casos:
\begin{enumerate}
  \item La paraula no és al diccionari.
  \item La paraula és al diccionari
    \begin{enumerate}
    \item i és també prefix de la següent (o següents) paraules, i podem trobar una d'aquestes paraules seguint la cerca des d'aquella posició.
    \item i és també prefix de la següent (o següents) paraules, però cap de les paraules que la contenen de prefix es poden trobar començant per aquella posició.
    \item i no és prefix de cap paraula.
    \end{enumerate}
\end{enumerate}
Per tant, una vegada s'ha trobat una paraula que es prefix s'ha de continuar buscant per assegurar-se de trobar totes les paraules més llargues que la tenen com a prefix en cas que hi siguin. En altres paraules, la cerca començant per aquella posició acaba quan al diccionari no hi ha cap de les paraules que s'hi generen en qualsevol direcció i sentit, incloent canvis de direcció en el transcurs de la paraula.\newline\par

En C++, les cadenes de text o \textit{string}, són definides com a vectors de caràcters. Per tant, podem veure el vector ordenat com un vector de vectors de caràcters o en altres paraules: com una matriu de caràcters. A la \textit{Table \ref{fig:sortedVector1}} es mostra un exemple visual d'un possible fragment del vector. \newline
\begin{table}[H]
\begin{center}
\begin{tabular}{l|l|l|l|l|}
\hline
\multicolumn{1}{|l|}{a} & a & a & a & a \\
\multicolumn{1}{|l|}{b} & c & c & i & i \\
\multicolumn{1}{|l|}{u} & a & c & r & r \\
\multicolumn{1}{|l|}{s} & b & i & e & e \\ \cline{1-1} \cline{4-4}
                        & i & o &   & s \\ \cline{2-3} \cline{5-5} 
\end{tabular}
\caption{Vector vist com una matriu de caràcters}
            \label{fig:sortedVector1}
\end{center}
\end{table}

Per recórrer aquesta "matriu" de caràcters utilitzarem tres variables. Les dos primeres: \textit{l} i \textit{r}, serveixen per buscar la primera i l'última ocurrència al vector de les paraules amb el prefix llegit a la sopa. La tercera variable s'anomena \textit{iterador} i serveix per saber en quina posició de la paraula s'ha de trobar cada nova lletra que llegim de la sopa. En cas que la paraula del \textit{Sorted Vector} tingui l'última lletra llegida de la sopa a la posició \textit{iterador}, podrem dir que la paraula tindrà almenys un prefix de mida \textit{iterador} igual que la concatenació de lletres llegides de la sopa. Aquestes són les paraules que haurem de procurar incloure dins el subvector [\textit{nl},\textit{nr}]. \newline\par
Aquesta explicació és molt més senzilla si l'acompanya l'exemple gràfic \textit{Table \ref{fig:sortedVector2}}, que mostra com els valors de \textit{l} i \textit{r} es mantenen a l'inici i fi del vector, ja que totes les paraules compleixen el prefix llegit de la sopa. En canvi, quan es llegeix un nou caràcter de la sopa, només hi ha dos paraules que tinguin la lletra buscada (en aquest cas, una i) a la posició \textit{iterador}. Per això canvien els valors de \textit{l} i \textit{r}.\newline

\begin{table}[H]
\begin{center}
\begin{tabular}{lllllll}
l                       &                        &                        &                        & r                      &          &                     \\ \cline{1-5}
\multicolumn{1}{|l|}{a} & \multicolumn{1}{l|}{a} & \multicolumn{1}{l|}{a} & \multicolumn{1}{l|}{a} & \multicolumn{1}{l|}{a} & iterador = 1 &                     \\
\multicolumn{1}{|l|}{b} & \multicolumn{1}{l|}{c} & \multicolumn{1}{l|}{c} & \multicolumn{1}{l|}{i} & \multicolumn{1}{l|}{i} &          &                     \\
\multicolumn{1}{|l|}{u} & \multicolumn{1}{l|}{a} & \multicolumn{1}{l|}{c} & \multicolumn{1}{l|}{r} & \multicolumn{1}{l|}{r} &          &                     \\
\multicolumn{1}{|l|}{s} & \multicolumn{1}{l|}{b} & \multicolumn{1}{l|}{i} & \multicolumn{1}{l|}{e} & \multicolumn{1}{l|}{e} &          &                     \\ \cline{1-1} \cline{4-4}
\multicolumn{1}{l|}{}   & \multicolumn{1}{l|}{i} & \multicolumn{1}{l|}{o} & \multicolumn{1}{l|}{}  & \multicolumn{1}{l|}{s} &          & \textbf{prefix: a} \\ \cline{2-3} \cline{5-5}
\end{tabular}
\end{center}
\begin{center}
\begin{tabular}{lllllll}
                        &                        &                        & l                      & r                      &          &                      \\ \cline{1-5}
\multicolumn{1}{|l|}{a} & \multicolumn{1}{l|}{a} & \multicolumn{1}{l|}{a} & \multicolumn{1}{l|}{a} & \multicolumn{1}{l|}{a} &          &                      \\
\multicolumn{1}{|l|}{b} & \multicolumn{1}{l|}{c} & \multicolumn{1}{l|}{c} & \multicolumn{1}{l|}{i} & \multicolumn{1}{l|}{i} & iterador = 2 &                      \\
\multicolumn{1}{|l|}{u} & \multicolumn{1}{l|}{a} & \multicolumn{1}{l|}{c} & \multicolumn{1}{l|}{r} & \multicolumn{1}{l|}{r} &          &                      \\
\multicolumn{1}{|l|}{s} & \multicolumn{1}{l|}{b} & \multicolumn{1}{l|}{i} & \multicolumn{1}{l|}{e} & \multicolumn{1}{l|}{e} &          &                      \\ \cline{1-1} \cline{4-4}
\multicolumn{1}{l|}{}   & \multicolumn{1}{l|}{i} & \multicolumn{1}{l|}{o} & \multicolumn{1}{l|}{}  & \multicolumn{1}{l|}{s} &          & \textbf{prefix: ai} \\ \cline{2-3} \cline{5-5}
\end{tabular}

            \caption{\textit{l} i \textit{r} limiten les paraules del vector que tenen com a prefix la cadena de caràcters que es va llegint}
            \label{fig:sortedVector2}
\end{center}
\end{table}

El procediment segueix i es van llegint nous caràcters. Imaginem que en aquest cas els caràcters llegits ens porten a una paraula que és dins del diccionari. Aquest podria ser, per exemple, el cas de la \textit{Table \ref{fig:sortedVector3}}. Veiem llavors que una paraula serà solució del diccionari si i només si al moment que es troba a la posició \textit{l}, el valor de \textit{iterador} és igual a la seva mida.  

\begin{table}[H]
\begin{center}

\begin{tabular}{lllllll}
                        &                        &                        & l                      & r                      &          &                        \\ \cline{1-5}
\multicolumn{1}{|l|}{a} & \multicolumn{1}{l|}{a} & \multicolumn{1}{l|}{a} & \multicolumn{1}{l|}{\textbf{a}} & \multicolumn{1}{l|}{a} &          &                        \\
\multicolumn{1}{|l|}{b} & \multicolumn{1}{l|}{c} & \multicolumn{1}{l|}{c} & \multicolumn{1}{l|}{\textbf{i}} & \multicolumn{1}{l|}{i} &          &                        \\
\multicolumn{1}{|l|}{u} & \multicolumn{1}{l|}{a} & \multicolumn{1}{l|}{c} & \multicolumn{1}{l|}{\textbf{r}} & \multicolumn{1}{l|}{r} &          &                        \\
\multicolumn{1}{|l|}{s} & \multicolumn{1}{l|}{b} & \multicolumn{1}{l|}{i} & \multicolumn{1}{l|}{\textbf{e}} & \multicolumn{1}{l|}{e} & iterador = 4 &                        \\ \cline{1-1} \cline{4-4}
\multicolumn{1}{l|}{}   & \multicolumn{1}{l|}{i} & \multicolumn{1}{l|}{o} & \multicolumn{1}{l|}{}  & \multicolumn{1}{l|}{s} &          & \textbf{prefix: aire} \\ \cline{2-3} \cline{5-5}
\end{tabular}
    \caption{Hem arribat a una solució}
            \label{fig:sortedVector3}
\end{center}
\end{table}

De totes maneres, tot i haver arribat a una solució, encara tenim la paraula "aires" inclosa al segment del vector [\textit{l},\textit{r}]. Per tant, hem de continuar llegint caràcters de la sopa per veure si trobem la lletra que ens falta. En cas que la trobem, \textit{l} es mourà una posició. Així doncs, \textit{l} marca la posició de la paraula "aires" quan la variable \textit{iterador} és igual a la mida de la paraula.

\begin{table}[H]
\begin{center}
\begin{tabular}{lllllll}
                        &                        &                        &                        & \begin{tabular}[c]{@{}l@{}}l\\ r\end{tabular} &              &                        \\ \cline{1-5}
\multicolumn{1}{|l|}{a} & \multicolumn{1}{l|}{a} & \multicolumn{1}{l|}{a} & \multicolumn{1}{l|}{a} & \multicolumn{1}{l|}{\textbf{a}}                        &              &                        \\
\multicolumn{1}{|l|}{b} & \multicolumn{1}{l|}{c} & \multicolumn{1}{l|}{c} & \multicolumn{1}{l|}{i} & \multicolumn{1}{l|}{\textbf{i}}                        &              &                        \\
\multicolumn{1}{|l|}{u} & \multicolumn{1}{l|}{a} & \multicolumn{1}{l|}{c} & \multicolumn{1}{l|}{r} & \multicolumn{1}{l|}{\textbf{r}}                        &              &                        \\
\multicolumn{1}{|l|}{s} & \multicolumn{1}{l|}{b} & \multicolumn{1}{l|}{i} & \multicolumn{1}{l|}{e} & \multicolumn{1}{l|}{\textbf{e}}                        &              &                        \\ \cline{1-1} \cline{4-4}
\multicolumn{1}{l|}{}   & \multicolumn{1}{l|}{i} & \multicolumn{1}{l|}{o} & \multicolumn{1}{l|}{}  & \multicolumn{1}{l|}{\textbf{s}}                        & iterador = 5 & \textbf{prefix: aires} \\ \cline{2-3} \cline{5-5}
\end{tabular}
    \caption{De nou, veiem que la condició per trobar una paraula es compleix.}
            \label{fig:sortedVector4}
\end{center}
\end{table}

Com que ja hem afegit l'única paraula que ens quedava dins el segment [\textit{l},\textit{r}], és hora de tirar enrere la cerca, llegir altres caràcters de la sopa, recuperar el prefix i les variables \textit{l} i \textit{r} anteriors. També s'ha de decrementar la variable \textit{iterador}. Una vegada s'ha retrocedit, ja podrem repetir el mateix procés cap a altres direccions fins no poder trobar més paraules.\newline

Per tant, s'està encarant un problema de cerca exhaustiva que hem de fer satisfent una restricció molt simple: que les paraules generades per la cerca pertanyin al diccionari o siguin prefix d'una o més paraules que hi pertanyin. Per aquesta raó s'ha decidit que el procediment \textit{buscarParaula} es programi amb un algorisme \textit{backtracking} recursiu, el codi del qual  es mostra al següent fragment.\newline

\begin{lstlisting}[language=C++]
void SortedVector:: buscarParaula(int i , int j, 
                                  vector<vector<bool>>& pos,
                                  int l, int r, int iterador,
                                  Sopa& s) {
    
    // Busquem la primera i ultima ocurrencia
    int  nl = firstOcurrence(l,r,s[i][j],iterador);
    int  nr = lastOcurrence(max(l,nl),r,s[i][j],iterador);

    // Cert si hem trobat paraules amb el prefix demanat
    if ((nl <= nr and nl != -1 and nr != -1)) {
        ++iterador;
        
        // Comprovem la condicio de solucio
        if (iterador == v[nl].size()) { 
            ++trobades[v[nl]]; // Afegim la paraula a la solucio
            // Incrementem nl (no necessitarem mes la paraula)
            ++nl; 
        }
        // Extenem la cerca cap a totes les direccions.
        int direccions_provades = 0, ni, nj;
        pos[i][j] = true;
        while (direccions_provades < 8) {
            ni = DIR[direccions_provades].first + i;
            nj = DIR[direccions_provades].second + j; 
            if (compleixLimits(s,ni,nj,s.size()) 
                                and not pos[ni][nj]) {
                buscarParaula(ni,nj,pos,nl,nr,iterador,s);
            }
                ++direccions_provades;
        }
        --iterador;
        pos[i][j] = false;
    }
}    
\end{lstlisting}


On, abans que res, es fa la cerca de la primera i última ocurrència de les paraules del diccionari que compleixen la condició i s'emmagatzemen a les variables \textit{nl} i \textit{nr} respectivament. Si \textit{nl} i \textit{nr} formen un interval correcte dins el vector, es busquen noves lletres de la sopa i es fa el mateix procés de cerca dins el subvector [\textit{nl},\textit{nr}].\newline\par 
El principal avantatge d'aquest tipus de cerca és que, per cada lletra nova que llegim, no caldrà cercar dins de tot el vector ordenat: podrem partir d'un subvector que ja sabem que conté totes les paraules amb prefix igual a les lletres que hem llegit anteriorment i des d'aquell punt, buscar l'interval que contingui les paraules el prefix de les quals són iguals a l'anterior, afegint-hi l'última lletra que hem llegit de la sopa.\newline\par
Per acabar d'entendre bé l'algorisme, mirarem com funcionen les cerques de la primera i l'última ocurrència.

\paragraph{\textit{firstOcurrence}} Com el lector ja haurà pogut deduir, es farà servir la cerca dicotòmica per dissenyar aquests algorismes. Cal entendre però, que una cerca dicotòmica normal busca només si l'element donat existeix al vector. En aquest cas però, volem cercar la primera ocurrència de l'element —un prefix— dins la sopa.
Recordem que la cerca dicotòmica descarta un segment del vector a través del càlcul del punt mig entre el  principi i el final del mateix. En aquest problema però, ens trobarem en els següents casos.
\begin{enumerate}
  \item No hi ha cap paraula amb el prefix demanat: S'acabarà la funció sense èxit.
  \item El punt mig conté el prefix buscat, però no és la primera ocurrència: Es tornarà a cridar la funció perquè cerqui la primera ocurrència dins el subvector que comença l'inici del vector i acaba a la posició anterior al punt mig.
  \item El punt mig conté el prefix buscat i és la primera ocurrència: Es retorna el punt mig.
  \item El punt mig conté un prefix posterior al buscat: Es repeteix la cerca descartant la segona meitat del vector.
  \item El punt mig conté un prefix anterior al buscat: Es repeteix la cerca descartant la primera  meitat.
  \item El punt mig conté una paraula de mida més petita que el prefix: Significa que la paraula no existeix o bé que està abans del punt mig. No pot estar situada posteriorment perquè les paraules després d'una paraula de mida inferior tindran un prefix diferent. Per tant la resolució del cas serà igual a la de l'apartat 2.
\end{enumerate}
També ens podem trobar als casos 4 i 5, i que finalment el prefix buscat no existeixi. Això no és problema ja que programem l'algorisme recursivament, i repetirem la cerca fins a veure que el prefix demanat no existeix. Això és el que podem veure reflectit al codi en C++ de la funció \textit{firstOcurrence}.\newline

\begin{lstlisting}[language=C++]
int SortedVector::firstOcurrence(int l, int r, const char& c,
                                                    int iterador) {
    // Cas 1
    if (l > r) return -1;
    int m = (l + r) / 2;
    
    if (v[m].size() > iterador) { 
        if  (v[m][iterador] == c) {
            if (l != m and (m != 0 and v[m-1].size() > iterador
                                        and v[m-1][iterador] == c)) {
                // Cas 2
                return firstOcurrence(l,m-1,c,iterador);
            } 
            else // Cas 3
                return m;
        }
        else if  (v[m][iterador] < c) // Cas 4
            return firstOcurrence(m+1,r,c,iterador);
            
        else // Cas 5
            return firstOcurrence(l,m-1,c,iterador);
    }
    else // Cas 6
        return firstOcurrence(l,m-1,c,iterador);

}
\end{lstlisting}

Com a nota final, cal recordar que dins el codi de la funció hi ha condicions addicionals perquè no totes les paraules del diccionari tenen la mateixa mida. Per això hem de comprovar que a l'accedir a una posició d'una paraula aquesta sigui vàlida. 

\paragraph{\textit{lastOcurrence}} Té una implementació semblant a \textit{firstOcurrence}, l'únic que ara necessitarem trobar l'última ocurrència d'un prefix enlloc de la primera. Dins el procés ens podem trobar amb els següents casos, que són molt semblants als de la funció anterior:

\begin{enumerate}
  \item No hi ha cap paraula amb el prefix demanat: S'acabarà la funció sense èxit.
  \item El punt mig conté el prefix buscat, però no és l'última ocurrència: Es tornarà a cridar la funció perquè cerqui l'última ocurrència dins el subvector que comença al punt mig i acaba al final del vector.
  \item El punt mig conté el prefix buscat i és l'última ocurrència: Es retorna el punt mig.
  \item El punt mig conté un prefix posterior al buscat: Es repeteix la cerca descartant la segona meitat del vector.
  \item El punt mig conté un prefix anterior al buscat: Es repeteix la cerca descartant la primera  meitat.
  \item El punt mig conté una paraula de mida més petita que el prefix: Significa que la paraula no existeix o bé que està abans del punt mig. No pot estar situada posteriorment perquè les paraules després d'una paraula de mida inferior tindran un prefix diferent.
\end{enumerate}

Implementat, i afegint-hi condicions addicionals per comprovar que no accedim a posicions d'un caràcter que no existeixen, queda així: \newline
\begin{lstlisting}[language=C++]
int SortedVector::lastOcurrence(int l, int r, const char& c,
                                                int iterador) {
    // Cas 1    
    if (l > r) return -1;
    int m = (l + r) / 2;
   
    if (v[m].size() > iterador) {
        if (v[m][iterador] == c) {
            if (r == m or (m == (v.size() - 1) ) 
                        or (c != v[m+1][iterador]) ) // Cas 3
                return m;
            else // Cas 2
                return lastOcurrence(m+1,r,c,iterador);
        }
        else if (v[m][iterador] < c) // Cas 4
            return lastOcurrence(m+1,r,c,iterador);
        else // Cas 5
            return lastOcurrence(l,m-1,c,iterador);
    }
    else // Cas 6
        return lastOcurrence(l,m-1,c,iterador);   
}
\end{lstlisting}
%%%%%%%%%%%%%%%%%%%%%%%%%%%%%%%%%%%%%%%%%%%%%%%%%%%%%%%%%%%%%%%%%%%%%%%%%%%%%%%%%%%%%%%%%%%%%%%%%%%%%%%%%%%%%%
\clearpage
\section{\textit{Trie}}
Una de les quatre maneres que tenim per resoldre la \textit{superSopa}, és guardant el diccionari en una estructura en forma de \textit{trie}. Aquesta estructura de dades presenta forma d'arbre i té diverses variants. 
\newline\par
Una d'aquestes variants, i la que fem servir en aquest projecte, és la \textit{ternary search tree}. Aquesta manera de guardar les dades és molt semblant a un \textit{binary search tree}, però en aquest cas contem amb un tercer fill. S'utilitza sobretot quan es tracta amb dades de tipus \textit{string}. 
\newline\par
Per acabar d'entendre el concepte de \textit{ternary search tree}, mirem un exemple: tenim les següents paraules i volem guardar-les en un arbre d'aquest tipus.\par
\begin{lstlisting}[language=C++]
    dicc = {nata, nas, nau}
\end{lstlisting}

Comencem per la primera paraula "nata", com l'arbre es troba buit, col·locarem el primer caràcter a l'arrel, el següent com a fill central, el següent com a fill central del anterior i així fins tenir-los tots col·locats.
\begin{center}
    \begin{forest}
        [n [a [t [a]]]]
    \end{forest}
\end{center}
Seguim amb la següent paraula "nas". Com l'arbre no està buit i segueix un determinat ordre, haurem de col·locar la paraula on li correspon. Per això, compararem el primer caràcter amb l'arrel. Tenim que tots dos caràcters són iguals, per tant, saltarem al fill central. Farem el mateix procediment, però enlloc d'observar el primer caràcter de "nas", observarem el segon. Tornem a obtenir el mateix resultat (a == a), per tant, saltem al fill central.
\newline\par
Ara estem en el node 't', farem el mateix procediment: observem el tercer caràcter de "nas", i ens troben en que són diferents. Concretament, 't' < 's', per tant, passarem al fill esquerre. Com aquest està buit, escriurem allà el caràcter 's' acabant així d'afegir la paraula.
\newline\par
Anem ara amb la última paraula "nau". Si ens fixem, seguirà el mateix procediment que la paraula anterior, excepte que al arribar al node 't' ens trobem en que 'u' < 't'. Per tant, anirem cap al fill dret i escriurem allà l'últim caràcter.
\newline\par
De manera que obtindríem un arbre com el de la figura.

\begin{center}
    \begin{forest}
        [n
            [a [t [s] [a] [u]] ]
        ]
    \end{forest}
\end{center}

En conclusió, aquest tipus d'estructura segueix un ordre amb l'objectiu de facilitar la cerca de paraules. Aquest ordre és (1) el fill esquerre ha de ser més petit que el pare (2) el fill dret ha de ser més gran que el pare i (3) el fill central ha de ser la continuació del pare.

\subsection{Especificació de la classe \textit{diccTrie}}

Per fer la implementació d'un diccionari en forma de \textit{trie}, hem dissenyat una serie de funcions i procediments que es poden trobar a la classe \textit{diccTrie.hh}. A continuació comentarem els atributs d'aquesta i les tres funcions que utilitzarem tant per emmagatzemar les paraules com per resoldre la sopa de lletres.

\subsubsection{Atributs privats}

Abans de començar amb la explicació de les funcionalitats, és important conèixer de dintre l'estructura amb la que estem treballant.

\begin{lstlisting}[language=C++]

    struct node_trie {
        string info;

        node_trie* esq;
        node_trie* dre;
        node_trie* cnt;

        bool finalparaula;
        
    };
\end{lstlisting}

Com hem comentat abans, cada node té tres fills, per tant, tindrem tres nodes que corresponen a cadascun d'ells. A més, també hi tenim un \textit{string} on hi gaurdarem un troç de paraula i un \textit{bool} que ens indicarà si aquest és l'últim troç que forma una paraula. 
\newline\par
Per acabar, també guardarem el primer node de l'arbre, l'anomenarem \textit{arrel}, i un \textit{map} per mantenir un registre de les paraules trobades i les vegades que hi apareixen a la sopa. Aquest \textit{map}, l'anomenarem \textit{results}.

\subsubsection{Funcionalitat \textit{afegir}}

Abans, quan hem explicat en que consistia un \textit{ternary search tree}, hem vist un exemple d'inserció, per lo que hem vist quina és la idea a seguir. Com treballem amb punters, la funció afegir cridara a una altra funció privada que recorrera l'arbre de forma rescursiva.

\begin{lstlisting}[language=C++]

    void afegirRec(node_trie* &n, const string& p, int i); 

    void TrieDictionary::afegir(const string& p) {
        afegirRec(arrel, p, 0)
    }
\end{lstlisting}

A la funció recursiva li passarem tres parametres: la paraula que volem afegir al diccionari, el primer node i un enter \textit{i} que ens indicarà quin caràcter falta colocar, és a dir, totes les lletres de la paraula \textit{p} que és trobin per davant de \textit{i}, hauran estat afegides a l'arbre.
\newline\par
Quan tractem amb funcions recursives hem de tenir clar quins casos ens podem trobar i, sobretot, quins d'ells finalitzen amb la recursivitat. Estudiem aquests casos:

\begin{itemize}
    \item \textbf{Cas base.} 
    \begin{enumerate}
        \item Es compleix que \textit{i == p.mida()}. \par
    Quan es comleix que \textit{i == p.mida()} significa que hem pogut col·locar totes les lletres que formen part de la paraula \textit{p}, per tant, hem acabat.
    \newline\par
    \end{enumerate}

    \item \textbf{Casos recursius.} 
    \begin{enumerate}
        \item L'element del node és més gran que el caracter \textit{i} de la paraula \textit{p}. \par
        Per tant, seguirem amb la crida recursiva passant com a node \textit{n->dre} i amb el mateix valor \textit{i} (ja que no hem col·locat cap lletra).
        \item L'element del node és més petit que el caràcter \textit{i} de la paraula \textit{p}. \par
        Molt semblant al punt 1. Seguirem amb la crida recursiva passant com a node \textit{n->esq} i amb el mateix valor \textit{i}.
        \item L'element del node és igual al caràcter \textit{i} de la paraula \textit{p}. \par
        En aquest cas, seguirem amb la crida recursiva passant com a node \textit{n->cnt}, però incrementarem una posició el nombre \textit{i}, ja que, encara que aquest caracter no l'haguem afegit, el reutilitzarem per la nova paraula.
        \item El node \textit{n} és \textit{null}. \par
        Aquest cas és molt semblant al cas base, però encara ens faltan més d'un caracter per afegir, és per això que la crida recursiva haurà de seguir, fins arribar a \textit{i == p.mida}, és a dir, el cas base. Farem la cirda recursiva passant el node \textit{n->cnt} i incrementant una posició \textit{i}.
    \end{enumerate}
\end{itemize}

De manera que la funció per afegir una paraula al diccionari queda de la següent manera:

\begin{lstlisting}[language=C++]

    void afegirRec(node_trie* &n, const string& p, int i) {
    
        string s(1, p[i]);

        if (n == nullptr) {
            node_trie* aux;
            aux = new node_trie;

            aux->info = s;
            aux->dre = nullptr;
            aux->esq = nullptr;
            aux->cnt = nullptr;
            aux->finalparaula = false;

            n = aux;
            ++i;

            if (i < p.size()) afegirRec(n->cnt, p, i);    // cas 4
            else n->finalparaula = true;                  // cas base
        }
        else if (n->info == s) {
            ++i;
            if (i < p.size()) afegirRec(n->cnt, p, i);    // cas 3
            else finalparaula = true;                     // cas base
        }
        else if (n->info > s) afegirRec(n->dre, p, i);    // cas 1
        else if (n->info < s) afegirRec(n->esq, p, i);    // cas 2
    }

\end{lstlisting}
\clearpage
\paragraph{Correctesa.} La justificació informal d'aquesta funció recursiva és la següent.
\begin{itemize}
    \item \textbf{Cas base.} \par
    Si tenim que \textit{i} és igual al nombre de lletres que conté \textit{p}, significa que ja hem colocat totes les lletres dintre l'arbre. Per tant, hem acabat.
    \item \textbf{Cas recursiu.}\par
    Si l'arbre és buit o el node en que treballem és buit, l'inicialitzarem amb la lletra \textit{i} de la paraula \textit{p}. Llavors, com ja no serà buit, podrem accedir sense error als seus subarbres. \par
    Si el node en que treballem no és buit, compararem la lletra que volem afegir amb el valor del node i, segons el seu valor, ens mourem cap a la dreta o esquerra.
    \item \textbf{Decreixament.}\par
    Si ens fixem en el codi, només incrementem la \textit{i} en aquelles crides que es desplacem al subarbre central. Això és perquè en aquests casos disminueix el nombre de lletres a afegir.\par
    També, a cada crida recursiva l'arbre es fa cada vegada més petit.
\end{itemize}

\paragraph{Cost.} Afegir una paraula de longitud \textit{k} en un \textit{ternary search tree} té un cost promig de \begin{math}
    \textit{O}(\log{n + k})
\end{math}.
Això depèn de si l'arbre es troba del tot equilibrat. És a dir, que té un nombre de nodes semblant al subarbre dret respecte al subarbre esquerra. Si no és així, el cas pitjor ens trobem amb un cost de \begin{math}
    \textit{O}(n + k)
\end{math}

\subsubsection{Funcionalitat \textit{simplificar}}
r
\subsubsection{Funcionalitat \textit{comprovar}}
c

\clearpage
\section{Filtre de Bloom}
\subsection{Introducció}
Un filtre de Bloom és una estructura de dades eficient en l'espai que permet comprovar si un element forma part d'un conjunt. No emmagatzema els elements en sí, només si en formen part, i no se'n poden eliminar elements.

Està compost per un vector d'\textit{m} bits, posats a 0, i una sèrie de \textit{k} funcions de \textit{hash} que generin distribucions aleatòries uniformes.

Per afegir-hi un element, es passa per les funcions de \textit{hash}, que retornen una sèrie d'índexs del vector. Els bits d'aquestes posicions del vector s'han d'activar (posar a 1).

Per comprovar si un element en forma part, també es passa per les funcions de \textit{hash}, es comproven els bits dels índexs obtinguts i, si tots estan activats, l'element forma part del conjunt.

Com a desavantatge, no es poden produir falsos positius. Si, per casualitat, els bits d'un element han estat activats per altres elements, el resultat serà que aquest element forma part del conjunt, cosa que no és certa. En un filtre de Bloom simple, com el que hem implementat, no es poden detectar, tot i que es pot dissenyar perquè el percentatge en sigui molt reduït. 

\subsection{Característiques}
El nombre de bits del vector i la quantitat de funcions de \textit{hash} necessàries es pot calcular amb dues senzilles fórmules matemàtiques.

Suposant que volem que funcioni correctament per un diccionari de 5000 elements (n = 5000) i tingui una ràtio de falsos positius de l'1\% (p = 0.01), tenim que:

\[m = \frac{-n * \ln(p)}{\ln(2)}\]

Així doncs, tindríem un vector de 47926 bits i 7 funcions de \textit{hash} diferents.

\subsection{Funcionament de la classe \textit{BloomFilterDictionary}}
El filtre de Bloom està implementat dins de la classe \textit{BloomFilterDictionary}, al fitxer \textit{diccBloomFilter.cc}.

La classe té dos atributs públics, un vector de booleans i la seva mida, un enter.

Tenim 7 funcions de \textit{hash} privades, de les quals 4 estan basades en un polinomi (en varia el coeficient), i 3 estan basades en una funció \textit{hash} d'enllaçament de fitxers objecte força comuna en C++.

En algun cas, degut a l'\textit{overflow}, el resultat de les funcions és negatiu. Això provocaria un error, car els índexs dels vectors no poden ser negatius. Per solucionar-ho, l'hem passat per la funció de valor absolut.

A més a més, disposem de dues funcions públiques per interactura amb el filtre de Bloom des de l'exterior de la classe: \textit{afegir} i \textit{comprovar}. 

Finalment, disposem de dos constructors. El constructor per defecte crea un filtre per un diccionari d'unes 5000 paraules. L'altre constructor rep per paràmetre la quantitat d'elements a guardar-hi, i calcula la mida del vector necessària. 
\subsection{Anàlisi del ràtio de falsos positius}

\clearpage
\section{Taula Hash}
\subsection{Funcionament de la classe}

\clearpage
\section{Classe \textit{SuperSopa}}

\clearpage
\section{Experimentació}
explicar com hem dissenyat els experiments (generar.cc i experiment\_vector.cc ...)

\clearpage
\section{Resultats}
\subsection{Filtre de Bloom: Falsos positius}

\clearpage
\section{Conclusions}

\clearpage
\section{Bibliografia}
\begin{enumerate}
    \item Campos, Javier (s.f.) \textit{2.4 Arboles digitales, tries y Patricia} [Diapositives].\url{http://webdiis.unizar.es/asignaturas/TAP/material/2.4.digitales.pdf }
    \item \textit{Implementación Trie en C - Insertar, Buscar y Eliminar.} (s. f.). Recuperat 12 d'octubre de 2022, de \url{https://www.techiedelight.com/es/trie-implementation-insert-search-delete/}
    \item \textit{Implementación en C++ de la estructura de datos Trie.} (s. f.). Recuperat 12 d'octubre de 2022, de \url{https://www.techiedelight.com/es/cpp-implementation-trie-data-structure/}
    \item \textit{10 formas de convertir un carácter en una cadena en C++.} (s. f.). Recuperat 13 d'octubre de 2022, de \url{https://www.techiedelight.com/es/convert-char-to-string-cpp/}
    \item \textit{Ternary Search Trees.} (s. f.). Dr. Dobb's. Recuperat 10 d'octubre de 2022, de \url{https://www.drdobbs.com/database/ternary-search-trees/184410528#}
    \item Wikipedia contributors. (2022, 12 septiembre). \textit{Trie}. Wikipedia. Recuperado 15 de octubre de 2022, de \url{https://en.wikipedia.org/wiki/Trie}
    \item \textit{Merge Sort (With Code in Python/C++/Java/C).} (s. f.). Recuperat 13 d'octubre de 2022, de \url{https://www.programiz.com/dsa/merge-sort}
    \item \textit{std::map - cppreference.com}. (s. f.). Recuperado 15 de octubre de 2022, de
    \url{https://en.cppreference.com/w/cpp/container/map}
    \item \textit{Create LaTeX tables online – TablesGenerator.com}. (s. f.). Recuperat 17 d'octubre de 2022, de \url{https://www.tablesgenerator.com/}

    \item \textit{Jutge.org - P84219\_en - First occurrence}. (s. f.). Recuperat 17 d'octubre de 2022, de \url{https://jutge.org/problems/P84219_en}

    \item Departament de Ciències de la Computació. (2017, octubre). \textit{Correctesa de Programes Iteratius i Programació Recursiva.} [Document]. \url{https://www.cs.upc.edu/pro2/data/uploads/it_rec_corr.pdf}

    \item Roura, Salvador (s. f.). \textit{Eficiència d'algorismes.} [Document]. \url{https://www.cs.upc.edu/eda/data/uploads/eficiencia.pdf}

    \item Rodrígez, Enric (s. f.). \textit{Correctesa i Anàlisi del Cost de l'Algorisme d'Euclides.} [Document]. \url{https://www.cs.upc.edu/eda/data/uploads/gcd.cat.pdf}
    
\end{enumerate}

Us deixo aquest enllaç per fer la bibliografia, \url{https://www.scribbr.es/citar/generador/apa/}.

\end{document}
